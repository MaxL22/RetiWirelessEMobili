% !TeX spellcheck = it_IT
\section{Teoria della Trasmissione}

Si vogliono trasmettere informazioni su un mezzo analogico: non perfetto, questo può introdurre \textbf{rumore}, \textbf{interferenze} o \textbf{attenuare} il segnale.

Un segnale può essere rappresentato: 
\begin{itemize}
    \item Nel dominio del \textbf{tempo}, vedendolo come un segnale periodico sinusoidale; rappresentazione "classica"
    
    \item Nel dominio delle \textbf{frequenze}, ogni segnale ragionevolmente periodico può essere scomposto in una serie di segnali periodici ($\sin$ e $\cos$), detti armoniche, ognuna con il relativo contributo rispetto al segnale originale
\end{itemize}

Quando tutte le armoniche sono multiple di una frequenza base, questa si chiama \textbf{frequenza fondamentale}.

\paragraph{Teorema del campionamento di Shannon:} Il segnale analogico va campionato, la frequenza di campionamento deve essere almeno il doppio della frequenza massima del segnale in ingresso. 

\paragraph{Relazione tra banda e data rate:} Per trasmettere perfettamente onde come composizioni di sinusoidali serverebbe banda infinita, quindi si usa un'approssimazione.

La quantità di informazioni inviata dipende dalla banda, per trasmetterne di più o si aumenta la banda o si peggiora l'approssimazione (usare meno armoniche).

\paragraph{Teorema di Nyquist sulla banda:} Dato un canale noise-free, il limite della quantità di informazioni è dato dalla formula
$$ C = 2 B \log_2 M $$
Dove
\begin{itemize}
    \item $B$ è la banda
    \item $M$ il numero di livelli del segnale (binario ne ha 2)
\end{itemize}

La capacità del canale aumenta con banda e numero di livelli del segnale, ma questo è solo un limite teorico in assenza di rumore.

\paragraph{Decibel:} Unità di misura del rapporto di potenze, in scala logaritmica:
$$ \left(\frac{P_1}{P_2}\right)_{dB} = 10 \cdot \log_{10} \left(\frac{P_1}{P_2}\right)$$

\paragraph{Rapporto segnale rumore SNR:} Per quantificare il rumore su un canale, e quindi il suo impatto sulla trasmissione
$$ SNR_{dB} = 10 \log_{10} \frac{P_\text{signal}}{P_\text{noise}}$$
Tanto più il rapporto è alto, tanto più il segnale si distingue dal rumore.

\paragraph{Formula della capacità di Shannon:} La capacità di un canale dipende anche dal rumore
$$ C = B \log_2 (1 + SNR) $$
Questa formula fornisce una massima teorica per la trasmissione senza errori sul canale.

Partendo dalla capacità data da Shannon si può trovare il numero di livelli da usare con un dato canale tramite l'inverso della formula data dal teorema di Nyquist:
$$ M = 2^{\frac{C}{2B}}$$
Trasmettere un numero maggiore di livelli è inutile, il rumore sarebbe troppo alto.

\subsection{Multiplexing}

Con "multiplexing" si intende la capacità di far passare più comunicazione all'interno dello stesso canale.

\paragraph{Frequency Division Multiplexing FDM:} Con una banda larga, la si può dividere in sotto-bande, ognuna con una comunicazione diversa.

\paragraph{Time division Multiplexing TDM:} Se il data rate è molto superiore a quello richiesto da una singola trasmissione, si possono creare $n$ slot di tempo ciclici.

\subsection{Comunicazione Wireless}

Le comunicazione wireless non possono avvenire in banda base (spettro $[0,B]$): ci sarebbero interferenze, le antenne richieste sarebbero troppo grandi, ogni range di frequenza ha delle proprietà specifiche.

Si usa quindi la banda traslata, si sposta il range a $[f_c - B/2, f_c + B/2]$, ovvero la stessa banda attorno a una frequenza portante $f_c$.

Il trasmettitore deve quindi codificare e modulare (attorno a $f_c$) il segnale prima di inviarlo.

\paragraph{Simbolo:} Uno stato significativo del canale di comunicazione che persiste per un tempo prefissato. In generale un simbolo può contenere più bit.

\paragraph{Tipi di antenne:} Una antenna può essere: 
\begin{itemize}
    \item Omnidirezionale: potenza emessa ugualmente in tutte le direzioni
    
    \item Direzionale: il segnale si propaga principalmente in una sola direzione
\end{itemize}

Il gain è il rapporto tra l'intensità della radiazione in una direzione rispetto all'intensità che si avrebbe con un'antenna isotropica/omnidirezionale, misurato in $dBi$.

\paragraph{Propagazione onde radio:} Esistono diversi tipi di propagazione delle onde radio, sopra i $30MHz$ la più comune è la Line of Sight LoS, che richiede una linea diretta tra RX e TX.

Problemi con la trasmissione LoS: 
\begin{itemize}
    \item Path loss: attenuazione dovuta alla distanza e ambiente in cui il segnale si propaga; misurato in $dB$
    $$ \frac{P_t}{P_r} = \left(\frac{4 \pi f}{c}\right)^2 d^n$$
    dove $n$ dipende dall'ambiente in cui si propaga il segnale (free space con $n=2$)
    
    \item Rumore: disturbo che può distorcere il segnale
    
    \item Multipath: riflessi di un segnale giunti al RX dopo aver già ricevuto l'originale, dovuti a riflessioni, rifrazioni o scattering; i percorsi "alternativi" sono più lunghi, le riflessioni vengono ricevute dopo. Inter-Symbol-Interference ISI: simboli successivi devono tener conto del multipath dei segnali precedenti
    
    \item Effetto doppler: il segnale cambia a causa del movimento di RX o TX
\end{itemize}

\subsubsection{Modulazione e Codifica}

L'obiettivo di codifica e modulazione è quello di ottenere una forma d'onda con un determinato significato.

Per la codifica si può agire su ognuno dei 3 parametri di una sinusoidale: Amplitude Shift Keying ASK, Amplitude Shift Keying ASK, Amplitude Shift Keying ASK.

\paragraph{Differential Phase-Shift Keying (DPSK):} Accorgersi di una differenza è più facile che misurare un livello, si codifica l'1 con un cambio di fase.

\paragraph{Quadrature Phase-Shift Keying QPSK:} Si usa la fase per determinare i bit. Ci sono 4 fasi differenti ($90^\circ$ tra ognuna), di conseguenza 2 bit per simbolo.

\paragraph{Quadrature Amplitude Modulation QAM:} Oltre alla fase si modifica l'ampiezza. Più parametri permettono più valori e una costellazione più densa. Si mappa prima in fase, poi in frequenza.

\subsubsection{Bit Error Rate BER Curve}

La curva BER rappresenta la probabilità che un bit venga alterato, in funzione del rapporto tra la densità di energia del segnale e il livello di rumore.

La probabilità di sbagliare a leggere una codifica, diminuisce all'incrementare del SNR. Tanto più una codifica ha una costellazione di valori densa più è facile sbagliare un simbolo (i valori sono vicini).

\paragraph{Adaptive Modulation and Coding AMC:} Si può adattare la codifica in base alla qualità del canale; si definisce un lasso di tempo ogni quanto misurare la qualità del canale e adattare la codifica di conseguenza.

\subsubsection{Forward Error Correction}

Servono tecniche di correzione dell'errore, molto probabile in ambito wireless. Viene aggiunta ridondanza ai dati inviati: ogni sequenza di bit lunga $n$ diventa una codeword lunga $k$ ($k > n$) secondo una tabella prefissata, rendendo la trasmissione più resiliente agli errori.

In AMC va scelta anche la coding rate, ovvero la proporzione tra bit di dati e bit di ridondanza.

\subsubsection{Orthogonal Frequency Division Multiplexing OFDM}

Multiplexing usando una divisione in frequenza con bande ortogonali tra loro (separate dal reciproco della durata del simbolo), in quanto bande ortogonali non possono causare interferenze a vicenda.

Si ha una portante $f_0$ e tante sotto-portanti multiple di $f_b$, scelta come
$$ f_b = \frac{1}{T}$$
dove $T$ è la durata di un simbolo.

Questo metodo 
\begin{itemize}
    \item permette di usare più banda rispetto a FDM in quanto non più necessaria una guardia (banda libera) tra le frequenze
    
    \item è più robusto a interferenze che riguardano solo alcune subcarrier
    
    \item è più robusto ai problemi di multipath perché la distanza tra un simbolo e l'altro è maggiore (ridotta ISI)
\end{itemize}

\subsubsection{Spread Spectrum}

Trasmettere il segnale occupando deliberatamente uno spettro più ampio del necessario, in modo da rendere il segnale più robusto, nascondere il segnale e permettere accesso multiplo.

\paragraph{Frequency Hopping Spread Spectrum FHSS:} Si genera pseudo-casualmente un codice di spreading, usato per determinare quale frequenza usare all'interno di una certa ampiezza di banda. La frequenza viene cambiata ogni intervallo di tempo prestabilito.

Generalmente resistente a rumore e jamming (compromettere tutte le frequenze è più difficile), nasconde parzialmente il segnale (è difficile leggere il messaggio senza sapere il codice di spreading).

\paragraph{Direct Sequence Spread Spectrum DSSS:} Ogni bit di una sequenza viene rappresentato da un insieme di bit, ottenuti da una sequenza casuale. Ogni bit trasmesso, chiamati "chip", dura $1/n$ del tempo dei bit di informazione, dove $n$ è il numero di chip che compongono un bit. 

Per mantenere lo stesso data rate è necessaria $n$ volte la banda.

Generalmente, si invia uno \texttt{xor} tra bit di informazioni e sequenza di spread (si manda la sequenza originale per uno 0, invertita per 1).

\paragraph{Code Division Multiple Access CDMA:} Ogni bit della sequenza da inviare viene codificato con un codice lungo $k \geq 1$ chip, unico per ogni utente. L'idea è che tutti gli utenti possono parlare assieme e le informazioni rimangono distinguibili. I chip usano 1 e -1 (al posto dello 0).

Un 1 viene codificato tramite il codice stesso, 0 con l'inverso. Il ricevitore moltiplica i bit ricevuti con il relativo bit all'interno del codice: se la somma di tutti questi è positiva è stato trasmesso un 1, 0 altrimenti.

Usando un codice non corretto per decifrare il risultato della sommatoria sarebbe inconcludente (vicino allo 0); il valore risultante deve essere abbastanza alto.

Più è alto il numero di utenti, più lo spreading factor deve essere alto; può essere modulato in funzione del numero di utenti.

Questo sistema funziona bene se i vari segnali ricevuti hanno circa la stessa potenza. \textbf{Near-far problem:} utenti più lontani avranno potenza minore. La soluzione è modulare la potenza in base alla distanza.