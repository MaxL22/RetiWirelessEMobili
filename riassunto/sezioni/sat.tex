% !TeX spellcheck = it_IT
\section{Satellitare}

Parametri importanti per la comunicazione satellitare: 
\begin{itemize}
    \item \textbf{Piano orbitale:} "cerchio" secondo cui il satellite si muove
    
    \item \textbf{Angolo di Azimuth:} orientamento rispetto al nord geografico
    
    \item \textbf{Angolo di elevazione:} angolo rispetto all'orizzonte
    
    \item \textbf{Angolo di copertura:} angolo tridimensionale della superficie terrestre visibile dal satellite
    
    \item \textbf{Lunghezza fisica del link:} la distanza dal satellite, importante per i tempi di trasmissione e varia in base al punto dell'orbita
    
    \item \textbf{Attenuazione in funzione dell'angolo di elevazione:} l'angolo di elevazione determina lunghezza del link e di conseguenza l'attenuazione
\end{itemize}

\subsection{Orbite}

\paragraph{Geostationary Earth Orbit GEO:} Orbita geostazionaria
\begin{itemize}
    \item periodo 24h 
    
    \item visibilità permanente
    
    \item angolo di elevazione fisso
\end{itemize}

Circa 35786km di altezza, ruota alla stessa velocità della Terra, quindi sembra fisso, permettendo di usare antenne fisse, allo stesso tempo avendo una copertura molto elevata. 

Di contro, alto delay, bassa qualità del segnale (molto lontano).

\paragraph{Low Earth Orbit LEO:} Orbita più vicina alla terra
\begin{itemize}
    \item periodo 1.5-2h
    
    \item visibilità 15/20 minuti per passaggio
    
    \item copertura 6000km
\end{itemize}

Rilevanti in particolare per applicazioni di comunicazione, hanno ridotto delay e RTT, necessitano di minore potenza di trasmissione. Posizionati a quota tra 160 e 2000km, sorvolano aree diverse a ogni passaggio, con una copertura relativamente limitata.

Il basso delay viene al costo di una limitata copertura e relativa necessità di grandi costellazioni, assieme a una gestione più complessa per quanto riguarda tracciamento, handover e comunicazione tra satelliti.

\paragraph{Medium Earth Orbit MEO:} Via di mezzo:
\begin{itemize}
    \item periodo 5-10h
    
    \item visibilità 2-8 ore per passaggio
    
    \item copertura 12000-15000km
\end{itemize}

Richiedono maggiore potenza e hanno RTT più lungo (decine di $ms$) dei LEO, ma comunque significativamente minore dei GEO. Usati da GNSS.

\subsection{Architettura}

Composta da 3 segmenti: 
\begin{itemize}
    \item \textbf{Space segment:} i satelliti e le relative costellazioni, possibili anche link inter-satellitari
    
    \item \textbf{Ground segment:} la "parte di controllo" del sistema, include gateway e tutto ciò che concerne le stazioni di terra per il controllo
    
    \item \textbf{User segment:} utilizzatori dei servizi, mobili o fixed
\end{itemize}

\subsection{Topologie di Rete}

\paragraph{Punto-punto:} Il satellite è solo ponte radio; due stazioni non si vedono e usano il satellite per comunicare. Permette una copertura maggiore rispetto alla rete terrestre, con un'elevata banda, ma delay e potenza elevati.

\paragraph{Broadcast:} Si sfrutta l'area coperta dal satellite, un solo satellite può trasmettere a molti ricevitori.

\paragraph{Mesh:} Link tra stazioni usando il satellite come ripetitore, non necessario un gateway.

\paragraph{Star:} Ogni stazione comunica passando dal satellite verso un gateway, il quale provvederà a smistare le comunicazioni.

\paragraph{Hybrid:} Combina star e mesdh, alcune comunicazioni passano dal gateway, ma si possono avere anche link diretti tra le stazioni 

\subsection{MAC}

I protocolli possono essere: 
\begin{itemize}
    \item Contention free: FDMA, CDMA, TDMA
    
    \item Contention-based random access: slotted RA, unslotted RA
\end{itemize}

\paragraph{TDMA:} Slot di tempo dedicato a ogni trasmittente. Per il downlink viene mandato un pacchetto unico, sta al ricevitore filtrare ciò che gli serve.

\paragraph{FDMA:} Minore efficienza spettrale, richiede di tenere conto dell'effetto doppler, soprattutto per orbite basse. Può essere combinata con TDMA.

\paragraph{CDMA:} Usato da GNSS. Soffre del near-far problem, molto rilevante per distanze \textit{satellitari}, poco rilevante per sistemi di posizionamento.

\paragraph{OFDMA:} Elevata sensibilità all'effetto doppler, richiede molta precisione.

\subsection{Integrated Satellite - Terrestrial Networks and 5G/6G NTN Technology}

Si vuole migliorare il servizio 5G/6G, anche in aree remote, con migliore continuità, affidabilità e scalabilità della rete.

\subsubsection{Architetture possibili}

Come si può integrare la rete satellitare alla rete operatore?

\paragraph{Relay:} Satellite per la comunicazione, relay a terra usato come ripetitore del segnale.

\paragraph{Backhaul:} Il satellite viene usato come rete di backhaul per reti difficili da raggiungere (gNodeB lontane dalla CN).

\paragraph{Direct Access:} La rete satellitare offre connettività direttamente ai dispositivi (device vede direttamente il satellite).

\subsubsection{Opzioni di integrazione}

Per integrare il satellite: 
\begin{itemize}
    \item \textbf{NTN Transparent Payload:} il satellite fa solo da relay, dialoga con le gNodeB, il gateway con le BS; si tratta di solo processore fisico
    
    \item \textbf{NTN Regenerative Payload:} Il satellite implementa lo stack di una gNodeB e ne svolge le funzionalità; implementa tutta la BS
    
    \item \textbf{NTN Regenerative Payload with Functional Split:} il satellite svolge solo le funzionalità del modulo distributed unit della gNodeB; i livelli di protocollo implementati dipendono dall'opzione di splitting scelta dall'operatore
\end{itemize}