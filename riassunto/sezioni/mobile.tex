% !TeX spellcheck = it_IT
\section{Mobile Network}

L'idea è molteplici trasmettitori, area divisa in celle, ogni cella servita da una Base Station BS, la quale contiene trasmettitore, ricevitore e unità di controllo.

Le celle devono coprire "bene" l'area e avere una disposizione uniforme (comunque considerando vincoli dovuti alla situazione reale).

\paragraph{Riuso delle frequenze:} Celle vicine non possono usare la stessa banda di frequenza. Tre soluzioni possibili: 
\begin{itemize}
    \item \textbf{Frequenze diverse tra celle vicine}: ogni cella ha la sua banda, servono più bande differenti (costano); usato da 2G
    
    \item Usare la \textbf{stessa frequenza ma tecniche di codifica} per evitare interferenze, come CDMA
    
    \item Usare \textbf{frequenze diverse ai bordi} delle celle, l'\textbf{intero spettro al centro}; permette banda maggiore ma richiede coordinamento tra le BS (da 4g e 5G)
\end{itemize}

\paragraph{Architettura:} Si divide in due macro aree: 
\begin{itemize}
    \item \textbf{Radio Access Network RAN:} stazioni e collegamento radio che forniscono connettività ai dispositivi mobili (parte wireless)
    
    \item \textbf{Core Network:} parte responsabile di gestione e controllo della comunicazione
\end{itemize}

In generale, esistono 2 tipologie di canali e traffico:
\begin{itemize}
    \item Canali di controllo: informazioni per la gestione delle operazioni; Control Plane
    
    \item Canali di traffico: voce e dati; Data Plane
\end{itemize}
La divisione tra i due piani è netta ed esplicita.

\subsection{Operazioni}

\paragraph{Inizializzazione:} Quando un dispositivo vuole iniziare una comunicazione con la rete cellulare: 
\begin{itemize}
    \item Disponibilità dei canali radio con BS: ascolta le trasmissioni broadcast delle BS per individuare la migliore e quali parametri usare
    
    \item Traffico di controllo per iniziare la comunicazione: viene inoltrata la richiesta di comunicazione fino al MTSO, parte della CN che si occupa di autenticare l'utente, verificare permesse e risorse
    
    \item Creazione dei collegamenti su data plane: se la fase di controllo va a buon fne, vengono allocati canali fisici e logici necessari per il traffico dati
\end{itemize}

\paragraph{Paging:} Quando \textit{qualcosa} deve raggiungere il dispositivo, è compito della rete trovarne la posizione quando necessario. MTSO contatta le BS per trovare il dispositivo.

\paragraph{Chiamata accettata:} I dati passano sempre per il CN, il dispositivo accetta la chiamata, MTSO crea un circuito e le BS impostano i canali radio data plane. 

\paragraph{Handoff:} Quando un dispositivo esce da una cella, deve collegarsi a un'altra. Le fasi per il cambio cella sono: decidere la nuova associazione, gestirla, riconfigurare i percorsi di comunicazione. 

Il tutto deve avvenire in maniera automatica e trasparente al dispositivo.

La procedure può essere decisa in due modi:
\begin{enumerate}
    \item solo dalla rete: in base al segnale in uplik
    
    \item coinvolgendo il dispositivo: anche in base a feedback sul downlink
\end{enumerate}

Vengono usate diverse metriche, ma il parametro principale è la potenza del segnale ricevuto. Possibili metodi per stabilire quando cambiare BS a cui il dispositivo è collegato: 
\begin{itemize}
    \item \textbf{Solo potenza relativa:} l'associazione viene fatta alla BS con il migliore segnale. Potrebbe portare a cambio continuo tra BS quando sui bordi delle celle e l'handover è costoso
    
    \item \textbf{Soglia di segnale:} L'handover viene fatto nel caso incui il segnale è peggiore di un'altra BS e allo stesso tempo sotto una certa soglia. Come si definiscono le soglie? (difficile)
    
    \item \textbf{Isteresi:} Per far partire l'handover deve esserci una differenza significativa di potenza, risolvendo il ping pong
\end{itemize}

La soluzione reale è combinare i metodi.

L'handover può essere:
\begin{itemize}
    \item \textbf{Hard}: dispositivo associato a una sola BS alla volta, cambio immediato di frequenza
    
    \item \textbf{Soft}: il dispositivo mantiene connettività con entrambe le BS e ne rilascia una quando il segnale è chiaramente dominante; occupa più risorse
\end{itemize}

\paragraph{Duplex:} Può essere gestito come: 
\begin{itemize}
    \item FDD: divisione in frequenza, minore delay, maggiori risorse richieste
    
    \item TDD: una sola frequenza e slot di tempo; maggiore ritardo dovuto all'attesa
\end{itemize}