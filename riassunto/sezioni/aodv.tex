% !TeX spellcheck = it_IT
\section{Ad Hoc Distance Vector Routing Protocol AODV}

Protocollo di routing per creare e riempire tabelle di instradamento, pensato per reti senza infrastrutture in cui ogni nodo è anche router. I percorsi vengono creati solo a necessità. 

Gli obiettivi principali sono: 
\begin{itemize}
    \item Gestione dinamica della rete ad hoc
    
    \item Auto inizializzante
    
    \item Loop free
    
    \item Risposta rapida alla richiesta di rotta e alla rottura di link/cambio di topologia
\end{itemize}

Le funzionalità offerte sono: 
\begin{itemize}
    \item Scoprire e costruire percorsi
    
    \item Mantenere percorsi in modalità soft-state (scadono se non aggiornati)
    
    \item Riconoscimento errori e cancellazione percorsi
\end{itemize}

Protocollo a livello di applicazione (UDP 654), al di sopra ci sono i messaggi AODV, sotto stack IP con PHY e LL qualsiasi.

\subsection{Tabelle di Routing}

Ogni nodo mantiene una tabella con destinazioni conosciute e next hop per arrivarci. Ogni entry della tabella contiene: 
\begin{itemize}
    \item IP Destinazione
    
    \item Sequence Number SN della destinazione: in una dinamica, ogni entry possiede un SN che codifica la "freschezza" dell'informazione. Si tratta di un valore per ogni nodo, modificabile solo dal nodo stesso: incrementato di 1 quando un nodo inizia o risponde a una ricerca di percorso; gli altri nodi possono aggiornare il SN quando vengono ricevute informazioni più aggiornate
    
    \item Flag di validità del SN (disabilita temporaneamente il percorso)
    
    \item Stato del percorso 
    
    \item Interfaccia di rete
    
    \item Hop count per arrivare alla destinazione
    
    \item Lista dei precursori: quali vicini utilizzano il router per arrivare alla destinazione
    
    \item Lifetime della entry: scadenza
\end{itemize}

\subsection{Route Request RREQ}

All'interno del pacchetto: Tipo 1. Ci sono delle flag: 
\begin{itemize}
    \item \texttt{D} Destination only: solo la destinazione può rispondere
    
    \item \texttt{U} Unknown SN: l'origine non conosce il SN della destinazione
    
    \item \texttt{G} Gratuitous: se risponde un nodo intermedio, questo deve creare il percorso bidirezionale
\end{itemize}

Oltre a questo contiene: 
\begin{itemize}
    \item Hop count: incrementato di 1 ogni inoltro, tiene traccia della distanza della richiesta
    
    \item RREQ ID: identificativo della richiesta, permette di riconoscere copie della stessa richiesta
    
    \item Destination IP Address e ultimo SN conosciuto
    
    \item Originator IP Address e SN
\end{itemize}

Una RREQ viene creata quando serve conoscere un percorso verso una destinazione. Per inviare:
\begin{itemize}
    \item L'originator incrementa RREQ ID e il proprio SN
    
    \item Setta i flag come necessario 
    
    \item Mantiene una copia per un tempo standard \texttt{PATH\_DISCOVERY\_TIME}, per evitare il riprocessamento e reinvio del pacchetto, se ricevuto nuovamente
\end{itemize}

\paragraph{Expanding Ring Search:} Per evitare di propagare "inutilmente" una RREQ in tutta la rete, il TTL dell'header IP della RREQ viene aumentato iterativamente, espandendo la ricerca man mano (valori di inizio, incremento e fine sono parametri). 

Senza conoscenze a priori si parte da un TTL standard, se ci sono informazioni riguardo la destinazione (percorso scaduto o interrotto), si usa l'hop count della vecchia entry come TTL.

\paragraph{Processamento e Inoltro:} Quando un nodo riceve una RREQ: 
\begin{itemize}
    \item Controlla se ha già visto la richiesta: \texttt{RREQ\_ID} e originator IP già noti
    
    \item Aggiorna il percorso verso l'originator: se il SN della richiesta è maggiore di quello posseduto, aggiorna la entry verso l'originator impostando come next hop il nodo da cui è arrivata la richiesta e hop count pari a quello dello RREQ
    
    \item Se non può rispondere con RREP, inoltra la richiesta in broadcast (livello IP) ai vicini incrementando di 1 l'hop count e ponendo il SN della DST al massimo tra quello della RREQ e quello presente all'interno della propria routing table
\end{itemize}

Il propagarsi di una richiesta permette ai nodi che la ricevono di costruire un percorso verso l'originator.

\subsection{Route Reply RREP}

Tipo 2, le flag sono:
\begin{itemize}
    \item \texttt{A}: richiede ack, per prevenire link non affidabili
\end{itemize}

Inoltre contiene:
\begin{itemize}
    \item prefix size: usato per subnet, indica la lunghezza del prefisso di rete dell'IP dst
    
    \item Destination IP Address, chi ha generato la risposta
    
    \item Destination SN
    
    \item Lifetime, in ms, tempo di validità della risposta
\end{itemize}

\paragraph{Creazione:} Chi può generare la risposta: 
\begin{enumerate}
    \item La destinazione: quando riceve la RREQ
    \begin{itemize}
        \item Incrementa il proprio SN di 1
        
        \item Hop count = 0
        
        \item Aggiorna lista dei precursori 
        
        \item Imposta il campo lifetime, secondo parametro
        
        \item Invia RREP lungo il percorso reverse in unicast 
        
        \item Drop della RREQ
    \end{itemize}
    
    \item Un nodo intermedio: con delle condizioni:
    \begin{enumerate}
        \item Avere una entry con percorso valido per la destinazione
        
        \item Flag \texttt{D == 0}
        
        \item DST SN della entry $\geq$ DST SN della RREQ
    \end{enumerate}
    Se tutte soddisfatte:
    \begin{itemize}
        \item Hop count = hop count della entry
        
        \item Aggiorna la lista dei precursori
        
        \item Imposta campo lifetime, pari a quello della entry
        
        \item Invia RREP lungo il percorso reverse in unicast
        
        \item Drop della RREQ
        
        \item Se flag \texttt{G == 1}, allora invia una RREP anche alla destinazione
    \end{itemize}
\end{enumerate}

\paragraph{Processamento e Inoltro:} Quando un nodo riceve un messaggio RREP
\begin{itemize}
    \item Aggiorna la entry nella tabella se: la corrente non è valida, DST SN della RREP maggiore di quello della entry o il numero di hop è minore
    
    \item Aggiorna
    \begin{itemize}
        \item Entry marcata come valida
        
        \item Next hop della entry = nodo da cui proviene RREP (DST della RREQ)
        
        \item RREP hop count += 1
        
        \item Lifetime della entry 
        
        \item Lista dei precursori
    \end{itemize}
\end{itemize}

\paragraph{RREP con flag Gratuitous:} Durante il ritorno della RREP, ogni nodo intermedio crea il path da se stesso alla destinazione originale della RREQ. Per creare il percorso bidirezionale anche tra nodo che ha risposto e destinazione serve la flag \texttt{G}. 

Se un nodo intermedio riceve RREQ con flag \texttt{G} alzata, deve "costruire" anche il percorso verso la destinazione della RREQ. Il nodo invia quindi 2 RREP indipendenti: 
\begin{enumerate}
    \item Verso il nodo originator della RREQ
    
    \item Verso la destinazione della RREQ, con parametri atti a "simulare" che la destinazione abbia fatto una RREQ verso l'originator, per completare il path bidirezionale
\end{enumerate}

\subsection{Hello Message}

Messaggi broadcast con TTL=1, senza incremento di SN, per fornire informazioni riguardo la connettività ai propri vicini. 

RREP speciale con:
\begin{itemize}
    \item DST IP = IP del nodo stesso 
    
    \item DST SN = SN del nodo 
    
    \item Hop count = 0 
    
    \item Lifetime = dato da parametri \texttt{ALLOWED\_HELLO\_LOSS * HELLO\_INTERVAL}
\end{itemize}

Meccanismo facoltativo, usato se non si ricevono altri informazioni da un vicino.

\subsection{Route Error RERR}

Ogni nodo ha il compito di tenere traccia della connettività con i nodi indicati come "next hop" nella tabella di routing. Questa può essere controllata tramite qualsiasi pacchetto a livello di data link o rete (se ricevo informazioni il link funziona).

Quando un nodo \textbf{identifica un link interrotto/scaduto} parte di un percorso attivo: 
\begin{itemize}
    \item Vengono invalidati i percorsi esistenti 
    
    \item Identifica le destinazioni per le quali viene usato come next hop il link interrotto
    
    \item Determina quali vicini nella lista dei predecessori possono essere affetti dal problema e invia un Route Error RERR
\end{itemize}

Tipo 3, le flag: 
\begin{itemize}
    \item \texttt{N} No Delete: indica alla destinazione di non eliminare la entry, il percorso è stato riparato localmente
\end{itemize}

Altri campi:
\begin{itemize}
    \item Destination Count: indica il numero di destinazioni non più raggiungibili contenute nel messaggio
    
    \item Unreachable Destination IP Address
    
    \item Unreachable Destination SN
\end{itemize}

Quando viene inviato un messaggio RERR:
\begin{itemize}
    \item Link interrotto trovato mentre si prova a inoltrare un pacchetto DATA (era il next hop verso una destinazione)
    
    \item Ricezione di pacchetto DATA per una destinazione sconosciuta (l'originator del pacchetto usa informazioni troppo vecchie)
    
    \item Ricezione di un pacchetto RERR da un vicino per uno o più percorsi attivi
\end{itemize}

\paragraph{Processamento e Inoltro:} Quando un nodo riceve una RERR:
\begin{itemize}
    \item Marca come invalide le entry della destinazione indicate nel RERR 
    
    \item Ogni entry invalidata viene preservata per un tempo (parametro)
    
    \item Inoltra RERR ai predecessori 
\end{itemize}

\paragraph{Local Repair:} Se un nodo riceve un pacchetto per una destinazione "non troppo lontana" (distanza definita) il cui percorso è interrotto, allora il nodo prova a riparare il percorso tramite una RREQ con TTL tale da non raggiungere la sorgente del pacchetto DATA (hop count del percorso verso originator). 

Due possibili esiti del tentativo di riparazione: 
\begin{itemize}
    \item procedura fallisce: manda RERR
    
    \item trova un percorso alternativo: aggiorna la propria entry
\end{itemize}

Se il percorso trovato è più lungo del precedente, invia una RERR con flag \texttt{N} alzata, sta alla sorgente decidere come procedere (la nuova lunghezza è accettabile o serve una RREQ?).