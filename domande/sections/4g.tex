% !TeX spellcheck = it_IT
\section{4G LTE}

\begin{questions}
    %Non mi piace la risposta, ma non ho davvero molto da dire su questo
    \question Descrivere il funzionamento del protocollo GTP (GPRS Tunneling Protocol) utilizzato nella rete mobile. Inoltre, discutere le motivazioni che hanno portato all'introduzione di questo protocollo e quali problemi risolve.
    
    \begin{solution}
        Il protocollo GTP è stato introdotto per gestire in modo efficiente la mobilità dei dispositivi. Lo User Equipment UE è libero di muoversi all'interno della rete, potenzialmente cambiando il punto di accesso alla rete (eNodeB e S-GW), cambiare le tabelle di routing di conseguenza sarebbe troppo dispendioso. 
        
        L'indirizzo IP del dispositivo dovrebbe rimanere unico per tutta la sessione, quindi GTP aggiunge ai pacchetti un header contenente un identificativo della sessione: Tunnel ID TEID. 
        
        In questo modo la sessione dello UE è identificata all'interno della rete operatore tramite il TEID, l'handover, anche a livello di S-GW, può essere gestito in maniera trasparente all'utente e alla rete esterna, il nodo di destinazione deve solo aggiornare le associazioni del TEID.
    \end{solution}
    
    \question Descrivere i 3 principali approcci di riutilizzo delle bande di frequenza nelle reti cellulari per ridurre l'interferenza tra celle adiacenti.
    
    \begin{solution}
        I 3 approcci principali per il riutilizzo delle frequenze sono: 
        \begin{itemize}
            \item La più intuitiva è usare frequenze diverse per celle adiacenti, la gestione è semplice ma viene sprecata molta banda in quanto effettivamente ne viene usata solo una porzione per volta
            
            \item Per non consumare più banda, una soluzione è usare tecniche di codifica per evitare interferenze tra celle vicine, come ad esempio CDMA (ogni utente ha il suo codice univoco, le trasmissioni sono distinguibili grazie a quello)
            
            \item L'ultima soluzione è utilizzare l'intera banda disponibile al centro della cella e ai bordi usare frequenze diverse per celle adiacenti; porta a un migliore uso della banda ma richiede un coordinamento più sofisticato tra le BS, oltre che riuscire a posizionare in maniera abbastanza precisa i dispositivi all'interno della cella
        \end{itemize}
    \end{solution}
    
    \question Che vantaggi porta il protocollo SCTP rispetto a TCP all'interno del control-plane di 4G LTE?
    
    \begin{solution}
        SCTP ha diversi vantaggi all'interno per l'utilizzo all'interno del control-plane 4G, tra cui
        \begin{itemize}
            \item Ammette ordinamento parziale: in TCP, se un segmento viene perso, tutti quelli successivi non vengono inviati ai livelli superiori, ma in 4G si ha la necessità di inviare dati relativi a tante connessioni indipendenti, quindi si aggiunge a ogni pacchetto uno Stream ID e la perdita di un pacchetto è bloccante solo rispetto ad altri pacchetti con lo stesso Stream ID
            
            \item SCTP è message oriented, al contrario di TCP che è stream oriented e richiede di inserire marcatori per delimitare i singoli messaggi
            
            \item SCTP permette il multi-homing: sorgente e destinazione della connessione non devono obbligatoriamente essere indirizzi singoli, possono essere multipli
        \end{itemize}
    \end{solution}
    
    \question In LTE 4G esistono due tipologie di handover: Seamless e Lossless. Come funzionano e in quali casi vengono usati.
    
    \begin{solution}
        In 4G LTE esiste solo hard handover (il dispositivo non è mai collegato a più BS contemporaneamente), ma gli handover si possono distinguere in due classi: 
        \begin{itemize}
            \item Seamless handover: latenza minore, ha come obiettivo far percepire il meno possibile l'handover, anche a costo di un (breve) buco nella connessione; viene usato per applicazioni real time, ad esempio traffico VoIP
            
            \item Lossless handover: maggiore latenza, ma ha come obiettivo non perdere nessun pacchetto; viene usato quando la ritrasmissione sarebbe troppo lenta, come ad esempio per traffico HTTP/FTP. Per evitare di perdere pacchetti in downlink, il flusso in download continuerà a mandare pacchetti alla vecchia BS, che li inoltrerà tramite interfaccia X2 alla BS di destinazione, la quale farà il buffer dei dati fino al termine dell'handover
        \end{itemize}
    \end{solution}
    
    \question Dato il seguente sequence diagram della procedura di handover: 
    \begin{enumerate}
        \item Quale interfaccia di comunicazione viene utilizzata?
        
        \item In quali casi è possibile?
        
        \item Descrivere i vari passi della procedura a partire dal punto 4 (la decisione di effettuare l'handover è stata presa)
    \end{enumerate}
    
    \begin{solution}
        L'interfaccia di comunicazione utilizzata è l'interfaccia X2,un canale di comunicazione diretto tra eNodeB.
        
        Questo tipo di handover è possibile solo nel caso il trasferimento sia effettuato tra celle appartenenti allo stesso MMU, i.e., cambia il bearer radio ma non il bearer S1.
        
        Dopo che è stata presa la decisione di handover: 
        \begin{itemize}
            \item La source eNodeB manda una richiesta di handover all'eNodeB di destinazione
            
            \item La eNodeB di destinazione alloca le risorse prima di rispondere con un ack alla richiesta
            
            \item Una volta che le risorse sono state allocate viene mandato il comando di handover dalla source eNodeB allo UE
            
            \item Viene trasferito lo stato della sessione in esecuzione sullo UE alla eNodeB di destinazione (per garantire continuità di servizio)
            
            \item Nel caso di lossless handover si ha una comunicazione su interfaccia X2 per trasferire i dati ricevuti in downlink dalla eNodeB sorgente durante il trasferimento
            
            \item Una volta che il trasferimento è completato, lo UE invia un messaggio di handover completato alla target eNodeB
            
            \item La target eNodeB invia una path switch request all'MME, ovvero una richiesta per cambiare l'indirizzamento dei dati relativi allo UE che è stato spostato; unica comunicazione con il CN, chiede di spostare il path dei dati
            
            \item L'MME risponde con un path switch ack
            
            \item Una volta terminato tutto, la target eNodeB invia un messaggio alla source eNodeB per rilasciare le risorse
        \end{itemize}
    \end{solution}
    
    \question Descrivere le funzionalità dei moduli principali (MME, SGW, PGW) in una rete LTE.
    
    \begin{solution}
        I moduli principali in una rete LTE sono: 
        \begin{itemize}
            \item Mobility Management Entity MME: si occupa di tutto il traffico di controllo e segnalazione all'interno della rete (tutto ciò che non è dati utente), come ad esempio gestione della mobilita, della tracking area, del paging, \dots
            
            \item Serving Gateway SGW: il nodo all'interno della CN che si occupa del traffico user; il punto di gestione per tutti i dati utente, con assegnato un gruppo di eNodeB
            
            \item Packet Data Network Gateway PDW: il punto di accesso tra rete LTE e le reti esterne, si occupa di cose come l'assegnamento dell'IP allo UE, garantisce le policy autorizzate per un certo utente, filtra i pacchetti per garantire QoS usando bearer diversi
        \end{itemize}
    \end{solution}
\end{questions}