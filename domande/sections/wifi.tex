% !TeX spellcheck = it_IT
\section{WiFi}

\begin{questions}
    \question Descrivere lo schema di accesso del protocollo 802.11 (Wi-Fi) CSMA/CA. Discutere, inoltre, che "accorgimento" viene aggiunto al meccanismo di backoff per evitare che una stazione attenda un tempo indefinito per accedere al canale.
    
    \begin{solution}
        Il sistema Carrier Sense Multiple Access/Collision Avoidance CSMA/CA richiede di aspettare "del tempo" prima di poter trasmettere. Definizioni delle unità di tempo: 
        \begin{itemize}
            \item Slot time: unità base di tempo, dipende dal trasmettitore fisico usato 
            
            \item SIFS: intervallo più breve, usato per messaggi ad alta priorità
            
            \item DIFS: intervallo più lungo, usato per messaggi a bassa priorità, pari a SIFS+2 slot time
            
            \item PIFS: intervallo di tempo intermedio, usato per servizi time-bounded, SIFS+slot time
        \end{itemize}
        
        Per trasmettere, un dispositivo: 
        \begin{itemize}
            \item Verifica che il canale sia libero tramite un Clear Channel Assessment CCA
            
            \item Ascolta per tempo DIFS il canale 
            
            \item fa un altro CCA 
        \end{itemize}
        Se il canale è risultato libero entrambe le volte, può cominciare a comunicare.
        
        Se il canale è occupato: il dispositivo aspetta il termine dell'altra trasmissione, per poi attendere tempo DIFS più un numero random di slot time (random backoff). Viene fatto carrier sense durante tutto il periodo di backoff, se il canale risulta libero può cominciare a trasmettere.
        
        Se durante il periodo di contesa il canale torna occupato, al turno successivo il dispositivo riprende il conteggio degli slot dal valore a cui era arrivato.
        
        Se necessario ack, il dispositivo attende tempo SIFS per riceverlo al termine della sua trasmissione, prima di presupporre che il frame sia stato corrotto e ritrasmettere.
    \end{solution}
    
    \question Descrivere, con un esempio, il problema del terminale nascosto in una rete Wi-Fi che adotta il protocollo CSMA/CA per l'accesso al canale radio condiviso. Quali modifiche vengono introdotte al protocollo CSMA/CA per risolvere questo problema? 
    
    \begin{solution}
        Se due terminali A e B, fuori dal rispettivo raggio di copertura, volessero trasmettere a B, vedrebbero il canale libero contemporaneamente, in quanto il meccanismo di carrier sense funziona solo se l'altro dispositivo che comincia a trasmettere può essere rilevato.
        
        Per risolvere il problema, il sender, dopo aver fatto carrier sense, invia una Request to Send RTS, contenete origine, destinazione e durata stimata della comunicazione da inviare, in questo modo tutti i terminali che ricevono la RTS senza esserne destinatari possono allocare un Network Allocation Vector NAV per la durata della comunicazione, in cui sanno che il canale verso la destinazione indicata sarà occupato.
        
        Il destinatario del messaggio risponde con un messaggio di Clear to Send CTS, anch'esso contenente origine, destinazione e durata stimata (rimanente) del messaggio. In questo modo anche i terminali all'interno del raggio di copertura del destinatario possono allocare un NAV per la durata della comunicazione.
    \end{solution}
    
    \question Data una rete WiFi che opera in moodalità DCF, descrivere la politica di backoff usata da una stazione che perde la contesa per l'accesso al canale.
    
    \begin{solution}
        Per ottenere l'accesso al canale in CSMA/CA una stazione deve effettuare due Clear Channel Assessment CCA a distanza DIFS. 
        
        Se durante uno di questi il canale è occupato, il dispositivo aspetta fino al termine della trasmissione in corso (facendo carrier sense), per poi aspettare tempo DIFS/PIFS/SIFS (in base alla priorità) e un numero di slot random (periodo di contesa). Se il canale è ancora libero dopo il periodo di contesa allora può cominciare a trasmettere.
        
        Se il canale torna occupato durante il periodo di contesa, il conteggio viene interrotto e ripreso al termine della trasmissione successiva (non riparte per evitare che una stazione aspetti indefinitivamente).
    \end{solution}
    
    \question Spiegare il processo di Downlink e Uplink tramite OFDMA in WiFi 6.
    
    \begin{solution}
        In WiFi 6 viene introdotto Orthogonal Frequency Division Multiple Access OFDMA per la suddivisone della banda tra più utenti. 
        
        In downlink, l'Access Point conosce la lista di destinatari e deve comunicare l'assegnamento delle risorse ai vari dispositivi, invia dunque una Multi-User Request to Send MU-RTS, per notificare ai dispositivi interessati l'assegnamento delle risorse e l'intento di comunicare.
        
        Dopo la MU-RTS, i dispositivi rispondono con una Clear to Send CTS, di seguito l'AP è libero di inviare in parallelo i dati a tutti i dispositivi.
        
        Se necessario, l'AP invia anche un Block Acknowledgment Request BAR e i dispositivi rispondono con un Block ack.
        
        Per l'uplink è necessario sincronizzare tutti i dispositivi, quindi: 
        \begin{itemize}
            \item L'AP invia un Buffer Status Report Poll BSRP, a cui i dispositivi rispondono con un Buffer Status Report BSR, usato per comunicare chi ha da trasmettere qualcosa
            
            \item L'AP invia la MU-RTS contenente l'allocazione delle risorse, i dispositivi rispondono con CTS
            
            \item Dopo un ulteriore trigger di sincronizzazione da parte dell'AP i dispositivi possono cominciare a trasmettere (padding se la trasmissione è troppo corta)
            
            \item Se necessario, l'AP invia un Multi-Station Block ack
        \end{itemize}
    \end{solution}
\end{questions}