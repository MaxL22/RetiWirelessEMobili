% !TeX spellcheck = it_IT
\section{WPAN}

\begin{questions}
    \question Nel protocollo 802.15.4 (Livello PHY e MAC di ZigBee) viene introdotto il concetto di "Duty-Cycle". Descrivere in cosa consiste e che vantaggi porta rispetto a 802.11 (Wi-Fi) per l'utilizzo in ambito IoT.
    
    \begin{solution}
        Il coordinatore di una rete ZigBee invia a intervalli regolari dei beacon, utilizzati per sincronizzare i dispositivi e organizzare i periodi di trasmissione. Il tempo tra un beacon e l'altro si divide in una prima parte di attività e una di inattività, per tutti i dispositivi. 
        
        Con duty cycle si intende l'alternarsi di periodi di attività e inattività, a radio spenta. Questo permette di ridurre significativamente i consumi rispetto ad altri standard, come 802.11, che richiedono di mantenere la radio sempre accesa. 
        
        In ambito IoT, in generale, i dati da trasmettere sono in quantità limitata e si dà la priorità a un basso consumo energetico. In questo caso tenere la radio spenta per la maggior parte del tempo può essere vantaggioso (i.e., non bisogna cambiare spesso la batteria ai sensori).
    \end{solution}
    
    \question Cosa si intende per Scatternet in una rete Bluetooth? Come viene gestita rispetto a una Piconet?
    
    \begin{solution}
        Una rete composta da un singolo master e diversi slave è definita Piconet. Se uno o più dispositivi appartengono a più piconet contemporaneamente si crea una Scatternet. 
        
        Le due (o più) piconet rimangono completamente indipendenti, clock, sequenza di hopping e slot di tempo completamente differenti, sta al dispositivo collegato a entrambe gestire la complessità aggiuntiva che questo porta.
        
        Sui 79 canali che Bluetooth usa per fare frequency hopping potrebbe capitare una sovrapposizione all'interno della scatternet, per evitare interferenze si usa CDMA (il master fornisce un codice ortogonale a ogni dispositivo).
    \end{solution}
    
    \question Descrivere il funzionamento della modalità di accesso al canale Slotted CSMA/CA utilizzata dal livello MAC dello standard 802.15.4 (usato da ZigBee e Thread).
    
    \begin{solution}
        La modalità CSMA/CA si fonda sull'invio di beacon da parte del coordinatore, usati per sincronizzare i dispositivi e organizzare i periodi di trasmissione.
        
        All'interno del superframe (divisione, logica, del tempo di trasmissione data dal coordinatore) sono presenti slot con diverso tipo di accesso: 
        \begin{itemize}
            \item Contention Access Period CAP: accesso al canale a contesa, usando CSMA/CA
            
            \item Contention Free Period CFP: intervallo di tempo per le comunicazioni con banda riservata
        \end{itemize}
        
        Nella modalità a contesa, un dispositivo, prima di trasmettere, verifica che il canale sia libero. 
        
        Per verificare la condizione del canale: 
        \begin{itemize}
            \item Viene deciso un valore casuale di backoff $[0, 2^{BE} - 1]$, dove il Backoff Exponent $BE$ è un parametro, e aspetta tale numero di slot (ognuno dura 20 simboli)
            
            \item Dopo l'attesa del numero casuale di slot vengono fatti un numero $CW$ (Contention Window, parametro) di CCA ravvicinati tra loro
            
            \item Se il canale è risultato libero in tutti i CCA il dispositivo comincia a trasmettere
            
            \item Se il canale è occupato, vengono incrementati di 1 $BE$ e $NB$ (Numero di Backoff, inizialmente 0), prima di ritentare; dopo 4 tentativi il livello MAC dichiara transmission failure
        \end{itemize}
        
        In qualsiasi caso, la trasmissione deve terminare nel numero di slot dedicati al CAP. Se il numero di slot non è sufficiente per la trasmissione viene rimandata al superframe dopo; similmente, se il conteggio durante il random backoff eccede il numero di slot, riprende al superframe successivo.
    \end{solution}
    
    \question In Bluetooth, la comunicazione all'interno della piconet utilizza FHSS, TDD e TDMA. Descrivere, usando uno schema temporale, come vengono utilizzate queste tre tecniche per la comunicazione tra master e slave all'interno di una Piconet Bluetooth.
    
    \begin{solution}
        In Bluetooth la comunicazione è divisa in slot da $625 \mu s$ e le tecniche utilizzate portano a: 
        \begin{itemize}
            \item Time Division Duplex TDD: la comunicazione alterna master e slave, negli slot pari comunica il master, in quelli dispari gli slave
            
            \item Time Division Multiple Access TDMA: ogni slot dedicato agli slave viene indirizzato a uno slave specifico
            
            \item Frequency Hopping Spread Spectrum FHSS: la banda viene divisa in 79 canali (40 per BLE) e la frequenza da utilizzare per ogni messaggio è determinata dal numero dello slot di tempo e da una sequenza pseudo-casuale condivisa dal master. Viene generata una sequenza casuale di frequenze e allo slot di tempo $n$ viene usata la $n$-esima frequenza. I messaggi possono durare 1, 3 o 5 slot, ma la frequenza utilizzata per un singolo messaggio non cambia durante la trasmissione
        \end{itemize}
    \end{solution}
    
    \question Descrivere, utilizzando un diagramma di sequenza, la procedura di Inquiry e Paging di Bluetooth. Si assuma che sia il dispositivo master e che il dispositivo slave siano nello stato iniziale di standby. Indicare inoltre in quale momento il dispositivo slave viene a conoscenza della specifica sequenza di frequency hopping del master.
    
    \begin{solution}
        In inquiry mode, il master invia periodicamente uno IAC packet su 32 canali standard. Gli slave ascoltano periodicamente su una delle 32 frequenze di wake up. Una volta che lo slave riesce a sentire una trasmissione del master: 
        \begin{itemize}
            \item Conoscendo il clock del master, fa un random backoff per poi rispondere con il proprio device address e classe
            
            \item Il master risponde con il Device Address Code DAC, sequenza per il frequency hopping e l'Active Member Address AMA
            
            \item Lo slave risponde con DAC e ack
        \end{itemize}
        
        La fase di inquiry permette di individuare i dispositivi, il paging serve a stabilire definitivamente la connessione, si ha uno scambio di pacchetti per la sincronizzazione per poi passare alla piconet, usando tutti i 79 canali e la sequenza di hopping completa.
    \end{solution}
    
    \question Descrivere le funzioni svolte dal Link Layer di BLE negli stati Advertising e Scanning, spiegando anche perché non è prevista una transizione di stato da Scanning a Connection.
    
    \begin{solution}
        Il Link Layer in BLE si occupa della trasmissione e ricezione dei pacchetti radio, anche sui canali di advertising.
        
        Nello stato di Advertising, uno slave si annuncia periodicamente, usando i canali di advertising. Se un master vuole raccogliere informazioni riguardanti i dispositivi vicini entra nello stato Scanning, in cui ascolta i pacchetti inviati sui canali di advertising.
        
        Non è prevista una transizione da Scanning a Connection in quanto lo stato di Scanning è solo per la raccolta dati, una volta che il master vuole collegarsi a un dispositivo entra nello stato Initiator.
    \end{solution}
    
    \question Spiegare come vengono gestiti più dispositivi Slave all'interno di una piconet Bluetooth. In particolare, descrivere come vengono gestiti duplex e multiple access?
    
    \begin{solution}
        All'interno di una piconet Bluetooth, più dispositivi slave vengono gestiti tramite i meccanismi di: 
        \begin{itemize}
            \item Time Division Duplex TDD: il master trasmette e i dispositivi ascoltano negli slot pari, viceversa negli slot dispari
            
            \item Time Division Multiple Access TDMA: il master decide, per ogni slot dedicato agli slave, quale di questi può comunicare
        \end{itemize}
    \end{solution}
    
    \question Che tipi di servizi supporta la baseband Bluetooth?
    
    \begin{solution}
        La baseband Blouetooth supporta due tipi di servizio: 
        \begin{itemize}
            \item Synchronous Connection Oriented Link SCO: connessione punto-punto con garanzie su bitrate minimo e delay, un canale riservato per casistiche real time; al massimo 3 in contemporanea
            
            \item Asynchronous Connectionless Link ACL: traffico best effort, occupa tutti gli slot rimanenti; permette qualità maggiore, ma senza garanzie
        \end{itemize}
    \end{solution}
    
    \question Descrivere lo schema di error control adottato da Bluetooth.
    
    \begin{solution}
        All'interno dei pacchetti Bluetooth sono presenti solo 2 bit usati per il controllo degli errori, ARQN e SEQN, rispettivamente un ack/nack e numero di sequenza per il pacchetto.
        
        In una trasmissione tipica: 
        \begin{itemize}
            \item Il master invia un pacchetto con SEQN $s$
            
            \item Lo slave risponde con ack e SEQN $s$
            
            \item Nello slot successivo si ripete, con SEQN $\neg s$
        \end{itemize}
        
        I possibili errori sono:
        \begin{itemize}
            \item Lo slave non riceve il pacchetto: risponde con un nack e SEQN $s$, il master sa che deve ritrasmettere grazie al nack
            
            \item Il master non riceve l'ack: in caso di ack perso il master ritrasmette sempre, lo slave si accorge della ritrasmissione in quanto il SEQN è pari a quello precedente
        \end{itemize}
    \end{solution}
    
    \question Cos'è un superframe in ZigBee? Come viene definito e come è composto.
    
    \begin{solution}
        A livello MAC si possono avere due modalità di trasferimento dati, sempre con CSMA/CA, ma possono essere slotted o unslotted.
        
        La modalità slotted utilizza i beacon: pacchetti di controllo, inviati dal coordinatore, contenenti informazioni per sincronizzare gli altri dispositivi, organizzare i periodi di trasmissione e gestire le trasmissioni di frame ancora pendenti. 
        
        L'organizzazione logica del tempo di comunicazione è detta "superframe", questa viene definita dal coordinatore e condivisa agli altri dispositivi attraverso i beacon.
        
        Il superframe viene diviso in slot con diverso tipo di accesso: 
        \begin{itemize}
            \item Contention Free Period CFP: periodo senza contesa, in cui il coordinatore garantisce la possibilità di comunicare ad alcuni dispositivi che lo necessitano
            
            \item Contention Access Period CAP: periodo a contesa, l'accesso al canale viene fatto usando CSMA/CA
        \end{itemize}
    \end{solution}
    
    \question Spiegare la procedura di accesso al canale tramite CSMA/CA in ZigBee. 
    
    \begin{solution}
        Nel periodo di trasmissione slotted CAP, l'accesso al canale viene regolato tramite CSMA/CA, quindi, per poter trasmettere, un dispositivo: 
        
        \begin{itemize}
            \item Aspetta un numero di slot di tempo random, nell'intervallo $[0, 2^{BE}]$, dove il Backoff Exponent $BE$ è un parametro
            
            \item Effettua un numero $CW$ di Clear Channel Assessment CCA ravvicinati, dove il valore di Contention Window $CW$ è un parametro
            
            \item Se il canale è risultato libero durante tutti i CCA può cominciare a trasmettere
            
            \item Se il canale è risultato occupato, aumenta il valore di $BE$ e $NB$ (Numero di Backoff, inizialmente 0) prima di riprovare
        \end{itemize}
        Se il conteggio degli slot durante la contesa viene interrotto da una trasmissione o dalla fine del periodo CAP, riprende dal valore a cui era arrivato all'occasione successiva, per evitare possibili attese infinite.
    \end{solution}
\end{questions}