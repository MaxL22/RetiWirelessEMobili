% !TeX spellcheck = it_IT
\section{AODV}

\begin{questions}
    \question In quali casi un nodo intermedio (non destinazione) che utilizza il protocollo AODV può rispondere con RREP alla ricezione di una RREQ? Che cosa deve fare il nodo se nel messaggio RREQ il flag Gratuitous è impostato a 1 (Assumendo che possa rispondere)? 
    
    \begin{solution}
        Un nodo intermedio può rispondere a una RREQ nel caso in cui
        \begin{itemize}
            \item Possiede un percorso valido verso la destinazione
            
            \item La flag \texttt{D} è a \texttt{0}
            
            \item Il SN della entry per il percorso non è minore del SN presente all'interno della RREQ
        \end{itemize}
        
        Per rispondere imposta hop count e lifetime pari a quelli presenti nella entry, invia la RREP sul percorso reverse in unicast, drop della RREQ. Se la flag \texttt{G} è alzata, invia una RREP anche verso la destinazione della RREQ originale, per indicare il percorso verso l'origine, quindi con hop count, lifetime, SN pari a quelli presenti nella RREQ, la destinazione è l'origine della RREQ e viceversa. In questo modo "simula" una richiesta di RREQ da parte della destinazione verso il nodo originale, in modo da completare il percorso tra i due.
    \end{solution}
    
    \question Si consideri il caso in cui un nodo "X", che utilizza AODV, riceva una RREQ dal nodo "Z", descrivibile come 
    \begin{center}
        \begin{tabular}{| c | c | c | c |}
            \hline
            \multicolumn{4}{| c |}{RREQ Ricevuta da Z} \\
            \hline
            Destination & C & Destination\_Only & 0 \\
            Destination SN & 100 & Gratuitous & 1 \\
            Originator & A & Hop count & 4 \\
            Originator SN & 200 & & \\
            \hline
        \end{tabular}
    \end{center}
    e in quel momento la sua tabella di routing contenga
    \begin{center}
        \begin{tabular}{| c | c | c | c |}
            \hline
            \multicolumn{4}{| c |}{Routing table X} \\
            \hline
            DST & NEXT & HOP & DST SN \\
            \hline
            C & G & 2 & 120 \\
            G & G & 1 & 99 \\
            Z & Z & 1 & 200 \\
            \hline
        \end{tabular}
    \end{center}
    
    Indicare
    \begin{itemize}
        \item il tipo e i campi del/dei messaggio/i che invia il nodo X (quello che riceve la RREQ)
        
        \item come cambia la tabella di routing del nodo X
    \end{itemize}
    
    \begin{solution}
        Dato che
        \begin{itemize}
            \item Il nodo X possiede un percorso per la destinazione della RREQ C
            
            \item Il SN del percorso è $\geq$ del SN indicato nella RREQ
            
            \item Il flag Destination\_only è a zero
        \end{itemize}
        il nodo X può rispondere alla RREQ, con una RREP contenente: 
        \begin{itemize}
            \item SN = 120, pari a quello della entry
            
            \item Hop count = 2, come nella entry
            
            \item Lifetime pari a quello della entry
        \end{itemize}
        
        Dato che il flag Gratuitous è alzato, invierà un'altra RREP verso la destinazione della RREQ, ovvero C, con
        \begin{itemize}
            \item Hop count = 4, come nella RREQ
            
            \item Destinazione = A, l'originator della RREQ
            
            \item SN della destinazione = 200, come nella RREQ
            
            \item Origine = C
            
            \item Lifetime pari alla entry verso l'origine della RREQ
        \end{itemize}
        
        "Simula" una RREQ da C verso A.
        
        Il nodo X aggiunge alla sua tabella di routing il percorso verso l'originator della richiesta A, con hop count 4 e next hop Z.
    \end{solution}
    
    \question Qual è la principale funzione del Sequence Number SN nel protocollo AODV? Quali nodi possono modificare il valore del SN?
    
    \begin{solution}
        La funzione principale del Squence Number SN nel protocollo Ad Hoc Distance Vector AODV è quello di contatore monotono crescente, usato per determinare la freschezza dell'informazione riguardante un nodo. Se più alto, l'informazione è più recente.
        
        Il SN di un nodo può essere modificato solo dal nodo stesso, in particolare viene incrementato di uno quando genera una RREQ o risponde a una RREQ destinata al nodo stesso.
    \end{solution}
    
    \question In AODV, un nodo Originator O riceve RREP in tempi diversi e da cammini diversi per la stessa RREQ. In base a quali parametri di RREP il nodo O sceglie il cammino migliore?
    
    \begin{solution}
        Un nodo Originator che ha inviato una RREQ, riceve più RREP in tempi diversi, i parametri per determinare il cammino migliore sono: 
        \begin{itemize}
            \item Numero di hop: all'interno della richiesta viene segnato il numero di inoltri effettuati, un percorso più breve è (solitamente) migliore
            
            \item Sequence Number SN: il SN codifica la freschezza dell'informazione, un SN più alto significa avere informazioni più recenti
        \end{itemize}
        
        Prima viene guardato il SN, poi l'hop count, il presupposto di AODV è che la rete sia dinamica, quindi un'informazione più recente è migliore. Nel caso entrambi i valori sono pari potrebbe essere guardato il lifetime rimanente.
    \end{solution}
    
    \question Descrivere le attività svolte da un nodo AODV quando riceve una RREQ.
    
    \begin{solution}
        Quando un nodo riceve una RREQ:
        \begin{itemize}
            \item Controlla se l'ID di richiesta e IP di origine sono già noti: se già visti, ignora la richiesta
            
            \item Aggiorna il percorso verso l'originator, se il SN dell'origine è maggiore della eventuale entry già presente
            
            \item Se non può rispondere, inoltra il messaggio ai vicini, incrementando l'hop count
            
            \item Se il flag destination only è a 0, e il nodo possiede una entry valida verso la destinazione con SN maggiore di quello all'interno della richiesta, allora può rispondere con una RREP
            
            \item Se ha risposto e il flag Gratuitous è alzato, invia una RREP anche al nodo destinazione della RREQ
        \end{itemize}
    \end{solution}
\end{questions}